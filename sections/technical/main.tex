\chapter{Программные средства технологических промышленных систем}



\section{Работа с нейронными сетями}

Развитие и применение нейронных сетей в задачах
динамической идентификации, неразрывно связано с
наличием мощных и гибких программных инструментов.
Эти инструменты предоставляют необходимые библиотеки
и фреймворки для создания архитектур нейронных
сетей, эффективного обучения на больших объемах
данных, в том числе с использованием специального
оборудования, например, графические процессоры GPU
или матричные процессоры TPU, и последующего
развертывания разработанных моделей. Выбор
соответствующего программного стека имеет
критическое значение для продуктивности
исследования, масштабируемости решений и возможности
воспроизведения результатов.

\subsection{Язык программирования}

Развитие инструментов для работы с нейронными сетями было неразрывно с
развитием языков и библиотек, нацеленных на высоконагруженные вычисления - C и
Fortran. Со временем, количество исследователей и пользователей увеличивалось и
развитие данных инструментов сместилось в область <<удобства>> пользования. В
частности, за последние гда выделилось множество языков, определивших базисный
инструментарий для работы с нейронными сетями. К языкам, активно используемых в
области машинного обучения и разработки нейронных сетей и составляющим этот
базис, можно отнести несколько языков программирования:

\begin{itemize}
  \item Python;
  \item MATLAB;
  \item R;
  \item C++;
  \item Julia.
\end{itemize}

\subsubsection{Python}
Python является интерпретируемым языком и особенно популярен в области науки о
данных и машинного обучения. Он обладает простым синтаксисом, обширной
экосистемой библиотек для научных вычислений и визуализации, а также большим
количеством высокопроизводительных фреймворков для глубокого обучения. Также,
язык поддерживает множество инструментов для создания графических
пользовательских интерфейсов. Его универсальность и легкость интеграции с
другими языками делают его привлекательным для полного цикла исследования — от
предобработки данных до построения и тестирования моделей. Также, несмотря на
интерпретируемость, большинство библиотек написаны на C, что позволяет
преодолеть проблемы с низкой производительностью интерпретатора. Кроме того,
решения на Python зачастую являются кросс-платформенными, что расширяет
возможности проектов для масштабирования и расширения. 

\subsubsection{MATLAB} 
MATLAB является пакетом для математических вычислений, включающий в себя
большое количество различных пакетов, предоставляет интегрированную среду
разработки и набор специализированных тулбоксов, в том числе для нейронных
сетей и глубокого обучения. Пакет реализует одноименный язык для написания
скриптов, основной особенностью которого является уклон в стороную простоты
написания и математической формализации.

Несмотря на широкий круг возможностей, он является коммерческим продуктом, что
может ограничивать доступность, и его экосистема библиотек, хотя и обширна в
своей нише, менее открыта и широка по сравнению с Python для самых передовых
исследований в области глубокого обучения. Кроме того, скрипты ограничены
окружением MATLAB и может быть переиспользованы обособлено.

\subsubsection{R}

R – это язык статистических вычислений и анализа данных с открытым исходным
кодом, созданный для работы с большими объемами данных, визуализации и
статистического моделирования. Имеет широкий спектр возможностей, включая
статистику данных, машинное обучение, визуализацию данных и т.д. Однако, язык
имеет неполную поддержку библиотек для работы с нейронными сетями и показывает
плохую производительность.

\subsubsection{C++} 
C++ является компилируемым языком общего назначения с высокой производительностью,
часто используемый для системного программирования и высоконагруженных
вычислений. Несмотря на высокую производительность и наличие библиотек, в
последние года их поддержка прекратилась и язык стал все реже использоваться в
проектах. Кроме того, сложность написания кода при работе с библиотеками и
высокому <<уровню входа>> в язык привела к оттоку разработчиков и переходу их
на Python.

\subsubsection{Julia} 
Julia – это высокоуровневый язык для научных вычислений, сочетающий скорость C++ с удобством Python и R. Язык имеет широкий список встроенных пакетов для научных вычислений и их ускорения, в том числе и с помощью параллелизации и графических ускорителей. Однако, язык относительно молодой и не имеет широкого стабильного инструментария для машинного обучения и в частности нейронных сетей. Кроме того, отсутствуют инструменты для раздработки оконных интерфейсов. 

Исходя из всех вышеописанных особенностей каждого из языков очевидно, что наиболее предпочтительным для разработки как модели для динамической идентификации, так и разработки для неё инструментов с оконным интерфейсом, наиболее всего подходит язык Python.

\subsection{Фреймворк}

Язык Python имеет множество библиотек, реализующих различные модели нейронных
сетей, а также алгоритмов машинного обучения. К таковым относятся как простые
библиотеки, предназначенные для изучения и обучения, так и промышленные
библиотеки, предназначенные для оптимальной и эффективной раброты с нейронными
сетями. Среди них особенно выделяются следующие библиотеки:

\begin{itemize}
  \item JAX;
  \item Tensorflow;
  \item PyTorch. 
\end{itemize}

\subsubsection{Tensorflow}

TensorFlow — фреймворк с открытым исходным кодом для построения и обучения
нейронных сетей. Поддерживает статические и динамические графы вычислений.
Интегрирован с библиотекой Keras, обеспечивающей более высокоуровневое API. 

Основным преимуществом данной библиотеки является реализация на C++,
обеспечивающая наибольшую скорость выполнения среди всех остальных библиотек.
Библиотека поддерживает все виды моделей и позволяет переносить и
переиспользовать обученные модели. 

\subsubsection{PyTorch}

PyTorch - это библиотека с открытым исходным кодом для работы с нейронными
сетями. Библиотека имеет реализацию на Python, в связи с чем её работа
обусловлена низкой скоростью. Библиотека имеет простой синтаксис и имеет малый
порог входа для начала работы. 

Библиотека имеет встроенные модули для работы с данными, а также встроенный
высокоуровневый API. Также, PyTorch поддерживает построение динамических графов
расчетов, что позволяет <<отлаживать>> обучение модели. 

Однако, в связи с малой скоростью работы, библиотека только используется в академических задачах. 

\subsubsection{JAX}

JAX является библиотекой для изучения нейронных сетей, объединяющая особенности
как PyTorch, так и Tensorflow, однако в связи с тем, что библиотека нова, имеет
плохую документацию и недостаточно описана. JAX имеет высокую
производительность, однако содержит меньше готовых компонентов и моделей для
разработки нейросетей. 

В соответствии с производительностью библиотек и возможностью интеграции с другими модулями, Tensorflow с модулем Keras является наиболее предпочтительным для работы в задаче динамической идентификации. 

\section{Инструменты создания оконных интерфейсов}

Инструментарий языка Python содержит широкий спектр библиотек, в том числе, обеспечивающих создание графических пользовательских интерфейсов. 
Наиболее часто используемыми библиотеками являются следующие

\begin{itemize}
  \item Tkinter;
  \item PyQt.
\end{itemize}


\subsection{Tkinter}

Tkinter - это простая встроенная библиотека Python
для создания базовых элементов GUI, часто
используемая для небольших проектов или быстрого
прототипирования. 

Библиотека включена в стандартную библиотеку Python
(начиная с версии 3.x), что делает его наиболее
доступным инструментом для создания
кроссплатформенных десктопных приложений. Tkinter
предоставляет объектно-ориентированный интерфейс для
работы с элементами управления (виджетами) и
поддерживает модель событийно-ориентированного
программирования.

Недостатком Tkinter является невозможность создания пользовательских
компонентов, что сильно ограничивает разработку сложных интерфейсов.

\subsection{PyQt}

PyQt - это кроссплатформенный GUI-фреймворк,
предоставляющий богатый набор виджетов и
инструментов для создания визуально привлекательных
и интерактивных приложений на Python.

Кроссплатформенность PyQt позволяет разрабатываемому
программному обеспечению работать на различных
операционных системах, обеспечивая более широкую
доступность. Он предлагает обширную коллекцию
готовых виджетов и инструментов для создания сложных
и удобных графических интерфейсов. PyQt тесно
интегрирован с Python, что позволяет разработчикам
беспрепятственно сочетать GUI с логикой приложения и
моделями машинного обучения.

PyQt выделяется как сильный выбор для разработки GUI благодаря сочетанию
кроссплатформенной совместимости, богатого набора функций для создания сложных
интерфейсов, бесшовной интеграции с Python и производительности, обеспечиваемой
его основой на языке C++.

