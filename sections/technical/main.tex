\chapter{Программные средства технологических промышленных систем}



\section{Работа с нейронными сетями}

Развитие и применение нейронных сетей в задачах
динамической идентификации, неразрывно связано с
наличием мощных и гибких программных инструментов.
Эти инструменты предоставляют необходимые библиотеки
и фреймворки для создания архитектур нейронных
сетей, эффективного обучения на больших объемах
данных, в том числе с использованием специального
оборудования, например, графические процессоры GPU
или матричные процессоры TPU, и последующего
развертывания разработанных моделей. Выбор
соответствующего программного стека имеет
критическое значение для продуктивности
исследования, масштабируемости решений и возможности
воспроизведения результатов.

\subsection{Язык программирования}

Развитие инструментов для работы с нейронными сетями было неразрывно с
развитием языков и библиотек, нацеленных на высоконагруженные вычисления - C и
Fortran. Со временем, количество исследователей и пользователей увеличивалось и
развитие данных инструментов сместилось в область <<удобства>> пользования. В
частности, за последние гда выделилось множество языков, определивших базисный
инструментарий для работы с нейронными сетями. К языкам, активно используемых в
области машинного обучения и разработки нейронных сетей и составляющим этот
базис, можно отнести несколько языков программирования:

\begin{itemize}
  \item Python;
  \item MATLAB;
  \item R;
  \item C++;
  \item Julia.
\end{itemize}

\subsubsection{Python}
Python является интерпретируемым языком и особенно популярен в области науки о
данных и машинного обучения. Он обладает простым синтаксисом, обширной
экосистемой библиотек для научных вычислений и визуализации, а также большим
количеством высокопроизводительных фреймворков для глубокого обучения. Также,
язык поддерживает множество инструментов для создания графических
пользовательских интерфейсов. Его универсальность и легкость интеграции с
другими языками делают его привлекательным для полного цикла исследования — от
предобработки данных до построения и тестирования моделей. Также, несмотря на
интерпретируемость, большинство библиотек написаны на C, что позволяет
преодолеть проблемы с низкой производительностью интерпретатора. Кроме того,
решения на Python зачастую являются кросс-платформенными, что расширяет
возможности проектов для масштабирования и расширения. 

\subsubsection{MATLAB} 
MATLAB является пакетом для математических вычислений, включающий в себя
большое количество различных пакетов, предоставляет интегрированную среду
разработки и набор специализированных тулбоксов, в том числе для нейронных
сетей и глубокого обучения. Пакет реализует одноименный язык для написания
скриптов, основной особенностью которого является уклон в стороную простоты
написания и математической формализации.

Несмотря на широкий круг возможностей, он является коммерческим продуктом, что
может ограничивать доступность, и его экосистема библиотек, хотя и обширна в
своей нише, менее открыта и широка по сравнению с Python для самых передовых
исследований в области глубокого обучения. Кроме того, скрипты ограничены
окружением MATLAB и может быть переиспользованы обособлено.

\subsubsection{R}

R – это язык статистических вычислений и анализа данных с открытым исходным
кодом, созданный для работы с большими объемами данных, визуализации и
статистического моделирования. Имеет широкий спектр возможностей, включая
статистику данных, машинное обучение, визуализацию данных и т.д. Однако, язык
имеет неполную поддержку библиотек для работы с нейронными сетями и показывает
плохую производительность.

\subsubsection{C++} 
C++ является компилируемым языком общего назначения с высокой производительностью,
часто используемый для системного программирования и высоконагруженных
вычислений. Несмотря на высокую производительность и наличие библиотек, в
последние года их поддержка прекратилась и язык стал все реже использоваться в
проектах. Кроме того, сложность написания кода при работе с библиотеками и
высокому <<уровню входа>> в язык привела к оттоку разработчиков и переходу их
на Python.

\subsubsection{Julia} 
Julia – это высокоуровневый язык для научных вычислений, сочетающий скорость C++ с удобством Python и R. Язык имеет широкий список встроенных пакетов для научных вычислений и их ускорения, в том числе и с помощью параллелизации и графических ускорителей. Однако, язык относительно молодой и не имеет широкого стабильного инструментария для машинного обучения и в частности нейронных сетей. Кроме того, отсутствуют инструменты для раздработки оконных интерфейсов. 

Исходя из всех вышеописанных особенностей каждого из языков очевидно, что наиболее предпочтительным для разработки как модели для динамической идентификации, так и разработки для неё инструментов с оконным интерфейсом, наиболее всего подходит язык Python.

\subsection{Фреймворк}

\section{Инструменты создания оконных интерфейсов}
