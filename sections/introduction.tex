\Introduction

Промышленность в настоящее время переживает период глубокой трансформации,
обусловленной повсеместным внедрением цифровых технологий. Такие концепции, как
цифровизация, IoT\Abbrev{IoT}{Internet of Things ""--- интернет вещей} и
Индустрия
4.0, становятся ключевыми факторами повышения эффективности, оптимизации
процессов и достижения новых уровней автоматизации в производственных
средах.\cite{iotoverview}
Четвертая промышленная революция, известная как Индустрия 4.0, характеризуется
интеллектуальным
производством, объединяющим как цифровые производственные технологии, так и
сетевые
коммуникации, компьютерные технологии и технологии автоматизации.

Эти изменения характеризуются повсеместным внедрением средств межмашинного
взаимодействия, автоматизации, машинного обучения и аналитики данных в реальном
времени, что приводит к слиянию физического производства и интеллектуальных
цифровых технологий.

Цифровая трансформация в промышленности предполагает оцифровку физических
активов и их интеграцию в цифровые экосистемы \cite{iotoverview}, что создает
потребность в системах, способных динамически идентифицировать и адаптироваться
к изменениям в этих сложных цифровых средах. Интеграция IoT и Индустрии 4.0
направлена на создание интеллектуальных фабрик и интеллектуальных
производственных сред путем объединения цифровых и физических процессов.

В контексте сложных промышленных систем динамическая идентификация играет важную роль в их оптимизации и управлении.

%Проверяем как у нас работают сокращения, обозначения и определения "---
%MAX, 
%\Abbrev{MAX}{Maximum ""--- максимальное значение параметра}
%API 
%\Abbrev{API}{application programming interface ""--- внешний интерфейс
%взаимодействия с приложением}
%с обратным прокси.
%\Define{Обратный прокси}{тип прокси-сервера, который ретранслирует}

