% Также можно использовать \Referat, как в оригинале
\begin{abstract}

    Отчет содержит \pageref{LastPage}\,стр.%
    \ifnum \totfig >0
    , \totfig~рис.%
    \fi
    \ifnum \tottab >0
    , \tottab~табл.%
    \fi
    %
    \ifnum \totbib >0
    , \totbib~источн.%
    \fi
    %
    \ifnum \totapp >0
    , \totapp~прил.%
    \else
    .%
    \fi


    Данная работа посвящена разработке и применению метода динамической
    идентификации сложных промышленных систем с использованием
    декомпозиционного подхода и нейронных сетей. Основной целью исследования
    является создание эффективного инструмента для построения цифровых
    двойников многокомпонентных систем, обеспечивающего высокую точность
    моделирования и упрощение процесса идентификации. Работа охватывает
    теоретические основы, разработку программного обеспечения и практическую
    апробацию предложенного метода на примере теплоэнергетического объекта —
    парового котла ГМ-50.
    \nocite{*}.

    Текст в документе носит совершенно абстрактный характер.

\end{abstract}

%%% Local Variables: 
%%% mode: latex
%%% TeX-master: "rpz"
%%% End: 
