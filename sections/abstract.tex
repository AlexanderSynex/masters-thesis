% Также можно использовать \Referat, как в оригинале
\begin{Referat}

    Отчет содержит \pageref{LastPage}\,стр.%
    \ifnum \totfig >0 , \totfig~рис.%
    \fi
    \ifnum \tottab >0 , \tottab~табл.%
    \fi
    %
    \ifnum \totbib >0 , \totbib~источн.%
    \fi
    %
    \ifnum \totapp >0 , \totapp~прил.%
    \else
    .%
    \fi

    Ключевые слова: идентификация систем, нейронные сети, моделирование
    процессов, программные средства идентификации, вычислительные системы.

    Тема выпускной квалификационной работы: <<Разработка алгоритмов и программ
    динамической идентификации энергетического объекта>>.

    Данная работа посвящена разработке и применению метода динамической
    идентификации сложных промышленных систем с использованием декомпозиционного
    подхода и нейронных сетей. Основной целью исследования является создание
    эффективного инструмента для построения цифровых двойников многокомпонентных
    систем, обеспечивающего высокую точность моделирования и упрощение процесса
    идентификации. Работа охватывает теоретические основы, разработку
    программного обеспечения и практическую апробацию предложенного метода на
    примере теплоэнергетического объекта — парового котла ГМ-50. 
    \nocite{*}

\end{Referat}

%%% Local Variables: %% mode: latex %% TeX-master: "rpz" %% End: 


\begin{ReferatEng}

    On \pageref{LastPage}\,pages%
    \ifnum \totfig >0 , \totfig~figures%
    \fi
    \ifnum \tottab >0 , \tottab~tables%
    \fi
    %
    \ifnum \totbib >0 , \totbib~sources%
    \fi
    %
    \ifnum \totapp >0 , \totapp~appendices%
    \else
    .%
    \fi

    Keywords: system identification, neural networks, process modeling,
    identification software tools, computational systems.

    The subject of the graduate qualification work is <<Development of
    algorithms and software for dynamic identification of an energy facility>>.

    This study focuses on the development and application of a dynamic
    identification method for complex industrial systems using a decompositional
    approach and neural networks. The primary objective of the research is to
    create an effective tool for building digital twins of multicomponent
    systems, ensuring high modeling accuracy and simplifying the identification
    process. The work covers theoretical foundations, software development, and
    practical validation of the proposed method, demonstrated on a thermal power
    facility—the GM-50 steam boiler.

\end{ReferatEng}