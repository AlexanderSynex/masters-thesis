\chapter{Динамическая идентификация систем}

Динамическая идентификация представляет собой процесс построения математических
моделей динамических систем на основе измерений их входных и выходных сигналов.
В динамической системе значения выходных сигналов зависят как от текущих
значений входных сигналов, так и от предшествующего поведения системы
\cite{bb:identification}. Типичные этапы динамической идентификации включают
следующие этапы: 

\begin{itemize}
  \item Определение и выделение исследуемых параметров системы;
  \item Формирование набора данных о работе исследуемой системы;
  \item Выбор метода идентификации;
  \item Разработка модели
  \item Оценка параметров и валидация полученой модели.
\end{itemize}

\section{Традиционные методы}

\subsection{Линейные модели}

\subsection{Нелинейные модели}

\section{Интеллектуальные методы}

\subsection{Алгоритмы машинного обучения}

\subsection{Нейронные сети}

