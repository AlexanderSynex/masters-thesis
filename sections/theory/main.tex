\chapter{Динамическая идентификация систем}

Динамическая идентификация представляет собой процесс построения математических
моделей динамических систем на основе измерений их входных и выходных сигналов.
В динамической системе значения выходных сигналов зависят как от текущих
значений входных сигналов, так и от предшествующего поведения системы
\cite{bb:identification}. Типичные этапы динамической идентификации включают
следующие этапы: 

\begin{enumerate}
  \item Определение и выделение исследуемых параметров системы;
  \item Формирование набора данных о работе исследуемой системы;
  \item Выбор метода идентификации;
  \item Разработка модели
  \item Оценка параметров и валидация полученой модели.
\end{enumerate}

Основная задача динамической идентификации состоит в выявлении неявных
закономерностей между параметрами модели, которые бы наилучшим образом
описывали наблюдаемое поведение системы. Такая модель может быть использована
для прогнозирования поведения системы при изменении входных параметров или для
оптимизации работы установки. 

Раздел математического моделирования, специализирующего на динамической
идентификации имеет широкий инструментарий и большое количество методов
построения цифровых двойников. В частности, все методы можно разделить на
следующие большие группы:

\begin{itemize}
  \item Параметрические;
  \item Непараметрические;
  \item Гибридные.
\end{itemize}

Параметрические методы включают в себя методы, предполагающие наличие
информации о структуре исследуемой модели. Для данного класса методов задача
сводится к определению параметров соотношений, описывающих модели. Данный класс
методов требует серьезного методологического анализа и учет мельчайших
особенностей работы системы. Данный класс методов зачастую сложно применять в
ситуациях, где система имеет большое количество компонентов и подвержена
внешним воздействиям, однако такие модели имеют наибольшую точность и наиболее
широко обхватывают все процессы, которым подвержена моделируемая система.
Непараметрические методы, в отличии от параметрических,
не требуют априорного знания о структуре модели и основывается лишь на данных о
работе системы. Данный класс методов основывается на построении
аппроксимационных моделей системы для определенных режимов работы.
Непараметрические методы применяются для описания сложных комплексных систем
для которых не требуется точности, но нужно общее поведение системы. Гибридные
методы же сочетают в себе особенности параметрических и непараметрических. 

С широким развитием интеллектуальных моделей, в частности, основанных на базе технологии нейронных сетей, классификацию методов можно разделить на традиционные и интеллектуальные методы.

\section{Традиционные методы}

\subsection{Линейные модели}

\subsection{Нелинейные модели}

\section{Интеллектуальные методы}

\subsection{Алгоритмы машинного обучения}

\subsection{Нейронные сети}

