\chapter{} {
\centerline{(основное)}
\section*{\centerline{Код системного модуля}}
% \blindtext \label{cha:appendix1}


\begin{lstlisting}
import tensorflow as tf
import keras
from keras import layers

import numpy as np
from sklearn.model_selection import train_test_split

from packages.Utils import DBStorage

class SystemRegressionModel():
    def __init__(self, inputs : int = 1, outputs : int = 1):
        super().__init__()
        if inputs < 0: inputs = 1
        if outputs < 0: outputs = 1
        self.__model = keras.Sequential([])
        self.__optimizer = keras.optimizers.Adam()
        self.__loss = keras.losses.MeanSquaredLogarithmicError
        self.__metrics = [ keras.metrics.RootMeanSquaredError() ]
        self.__inputs = inputs
        self.__outputs = outputs
        self.__layers = [ ]
        self.__accuracy = 0
        self.__rebuild()


    @property
    def inputs(self) -> int: return self.__inputs
    
    @inputs.setter
    def inputs(self, value : int): 
        if value > 0: 
            self.__inputs = value
        self.__rebuild()
    
    @property
    def outputs(self) -> int: return self.__outputs
    
    @outputs.setter
    def outputs(self, value : int): 
        if value > 0: 
            self.__outputs = value
        self.__rebuild()
    
    @property
    def model(self) -> keras.Sequential: return self.__model
    
    def __rebuild(self):
        self.__layers = [ max(self.__inputs, self.outputs) * 2 for _ in range(max(self.__inputs, self.__outputs)) ]
        layers_ = [layers.Input(shape=(self.__inputs,), name='Input'), 
                   layers.Dense(self.__inputs, activation=keras.activations.relu)]
        for i, layer in enumerate(self.__layers):
            layers_.append(layers.Dense(layer, activation=keras.activations.relu))
        layers_.append(layers.Dropout(0.2))
        layers_.append(layers.Dense(self.__outputs, activation=keras.activations.relu, name='Output'))
        self.__model = keras.Sequential(layers=layers_)
        
    def compile(self):
        self.model.compile(optimizer=self.__optimizer,
                           loss=self.__loss,
                           metrics=self.__metrics)
    
    @property
    def mse(self):
        return self.__accuracy
    
    def fit(self, input_keys : list, output_keys : list, epochs=500):
        inputs, outputs = [], []
        for signal in input_keys:
            x = DBStorage.find(signal)
            if not x.empty:
                inputs.append(x.to_list())
        for signal in output_keys:
            y = DBStorage.find(signal)
            if not y.empty:
                outputs.append(y.to_list())
        X = np.array(inputs).transpose()
        y = np.array(outputs).transpose()
        X_train, X_test, y_train, y_test = train_test_split(X, y, test_size=0.33, random_state=42)
        self.model.fit(X_train, y_train, epochs=epochs)
        self.__accuracy = self.model.evaluate(X_test, y_test)[1]from ..Connection import ConnectionManager

from .SystemRegressionModel import SystemRegressionModel

class System(object):
    def __init__(self, name : str = "", Inputs = [], Outputs = []):
        self.__name : str = name
        self.__Inputs = []
        self.__Outputs = []
        self.__weight = 0
        
        self.__model = SystemRegressionModel()
        self.add_inputs(Inputs)
        self.add_outputs(Outputs)
        
    
    @property
    def name(self):
        return self.__name
    
    @property
    def input_keys(self):
        return self.__Inputs
    
    @property
    def output_keys(self):
        return self.__Outputs
    
    @property
    def weight(self):
        return self.__weight
    
    @weight.setter
    def weight(self, weight):
        self.__weight = max(self.__weight, weight)
    
    @property
    def model_wrapper(self):
        return self.__model
    
    @property
    def model(self):
        return self.__model.model
    
    def add_input(self, name):
        ConnectionManager().get_instance(name)
        self.__Inputs.append(name)
        
        self.weight = ConnectionManager().get_instance(name).weight
        self.__model.inputs = len(self.__Inputs)
    
    
    def add_inputs(self, Inputs):
        for name in Inputs:
            self.add_input(name)
    
        
    def add_output(self, name):
        ConnectionManager().get_instance(name)
        self.__Outputs.append(name)
        ConnectionManager().get_instance(name).weight = self.__weight + 1
        self.__model.outputs = len(self.__Outputs)
    
        
    def add_outputs(self, Outputs):
        for name in Outputs:
            self.add_output(name)
    
    def fit(self, epochs=100):
        self.model_wrapper.fit(input_keys  = self.__Inputs, 
                               output_keys = self.__Outputs,
                               epochs=epochs)from ..Singleton import Singleton
from .System import System
from .SystemManager import SystemManager

from ..Connection import ConnectionManager, ConnectionDataWrapper

import pandas as pd

import json

class SystemDataWrapper(metaclass=Singleton):
    def to_dict(cls, system_name : str) -> dict:
        if not SystemManager().exists(system_name):
            return {}
        
        sys = SystemManager().get_instance(system_name)
        
        inputs = [name for name in sys.input_keys]        
        outputs = [name for name in sys.output_keys]
        
        return dict(name=system_name, inputs=inputs, outputs=outputs)

    
    def to_json(cls, system_name : str) -> str:
        if not SystemManager().exists(system_name):
            return None
        
        return json.dumps(cls.to_dict(system_name=system_name), indent=2, ensure_ascii=False)

    
    def all_to_dict(cls):
        if SystemManager().empty():
            return dict(signals=[],systems=[])
        
        signals=[]
        for name in ConnectionManager().get_keys():
            signals.append(ConnectionDataWrapper().to_dict(link_name=name))
        
        systems=[]
        for name in SystemManager().get_keys():
            systems.append(cls.to_dict(name))
        
        return dict(signals=signals, systems=systems)
    
    
    def all_to_json(cls) -> str:
        return json.dumps(cls.all_to_dict(), indent=2, ensure_ascii=False)
    
    
    def from_dict(cls, system_dict : dict):
        if not all(key in system_dict for key in ('name',
                                                  'inputs', 
                                                  'outputs')):
            return
        
        input_keys = system_dict['inputs']
        output_keys = system_dict['outputs']
        
        SystemManager().get_instance(name=system_dict['name'],
                                     Inputs=input_keys,
                                     Outputs=output_keys)
        
        ConnectionManager().rebase_internal_connections()
    
    
    def from_json(cls, path : str):
        with open(path) as file:
            raw_data = json.load(file)
            
            if not all(key in raw_data for key in ('systems', 'signals')):
                return
            
            for signal in raw_data['signals']:
                ConnectionDataWrapper().from_dict(signal)
            
            for system in raw_data['systems']:
                cls.from_dict(system)from .System import System
from ..Singleton import Singleton

from ..Connection import ConnectionManager

# Manager purposed to store names of created systems and grant their uniqueness
class SystemManager(metaclass=Singleton):
    _systems = {}

    def __add(cls, system):
        name = system.name
        if name not in cls._systems:
            cls._systems[name] = system
    
    def __get(cls, name):
        if name not in cls._systems:
            return None
        return cls._systems[name]
    
    def get_instance(cls, name : str = "", Inputs = [], Outputs = []):
        system = cls.__get(name)
        if system is None:
            system = System(name, Inputs=Inputs, Outputs=Outputs)
            
            for link_name in Inputs:
                if ConnectionManager().exists(link_name):
                    ConnectionManager().get_instance(link_name).add_to_system(name)
            
            for link_name in Outputs:
                if ConnectionManager().exists(link_name):
                    ConnectionManager().get_instance(link_name).add_from_system(name)
            
            cls.__add(system)
        return system
    
    def clear(cls):
        ConnectionManager().clear()
        cls._systems = {}
    
    def exists(cls, name):
        return cls.__get(name) != None
    
    def empty(cls):
        return not cls._systems
    
    # Return all keys
    def get_keys(cls):
        return list(cls._systems.keys())import numpy as np

from packages.SystemModule.Singleton import Singleton

from packages.SystemModule.System import SystemManager
from packages.SystemModule.Connection import Connection, ConnectionManager

from collections import deque

class SystemNeuralModeling(metaclass=Singleton):
    @classmethod
    def feed_forward_system(cls,system_name):
        if not SystemManager().exists(system_name):
            return
        system = SystemManager().get_instance(system_name)
            
        # Making input for current system
        x = np.zeros(len(system.input_keys))
        for i, input in enumerate(system.input_keys):
            if ConnectionManager().exists(input):
                x[i] = ConnectionManager().get_instance(input).value
        x = np.array([[xi] for xi in x]).reshape((1, -1))
        y = system.model(x)
        print(f"System={system_name}")
        for i, output in enumerate(system.output_keys):
            if ConnectionManager().exists(output):
                ConnectionManager().get_instance(output).value = y[0, i]
                print(f"\t{output}={ConnectionManager().get_instance(output).value}")
    
    
    @classmethod
    def feed_forward_from_signal(cls,signal_name):
        print(f"Here: {signal_name}")
        if not ConnectionManager().exists(signal_name):
            return
        visited = set()
        
        systems : set = ConnectionManager().get_instance(signal_name).to_systems
        next_systems = set()
        for system_name in systems:
            if not SystemManager().exists(system_name):
                continue
            system = SystemManager().get_instance(system_name)
            visited.add(system_name)
            SystemNeuralModeling.feed_forward_system(system_name)
            for output in system.output_keys:
                if not ConnectionManager().exists(output):
                        continue
                signal = ConnectionManager().get_instance(output)
                next_systems = next_systems.union(signal.to_systems)
        for system in next_systems:
            SystemNeuralModeling.feed_forward_system(system)
    
    @classmethod
    def system_next_systems(cls, system_name) -> set:
        if not SystemManager().exists(system_name):
            return set()
        
        system = SystemManager().get_instance(system_name)
        systems = set()
        for signal_name in system.output_keys:
            if not ConnectionManager().exists(signal_name):
                continue
            signal = ConnectionManager().get_instance(signal_name)
            systems = systems.union(signal.to_systems)
        return systems
    
    @classmethod
    def recalculate(cls):
        # find input signals
        inputs = set()
        for signal_name in ConnectionManager().get_keys():
            signal : Connection = ConnectionManager().get_instance(signal_name)
            if not signal.from_system and signal.to_systems:
                inputs.add(signal_name)
        
        # set of systems
        unique_systems = set()
        for signal_name in inputs:
            signal : Connection = ConnectionManager().get_instance(signal_name)
            for system_name in signal.to_systems:
                if (SystemManager().exists(system_name)):
                    unique_systems.add(system_name)
        
        systems = deque()
        for system in unique_systems:
            systems.append(system)
        
        visited : set = set()
        
        print(f"START ITERATING OVER: {systems}")
        
        while systems:
            system_name = systems.popleft()
            if system_name in visited:
                continue
            visited.add(system_name)
            print(f"System.{system_name}")
            cls.feed_forward_system(system_name=system_name)
            # feed forward
            neighbours : set = cls.system_next_systems(system_name=system_name)
            for system in neighbours:
                systems.append(system)
        
        # get output signals
        # find set of systems from outputs <- exit condition# Metaclass
class Singleton(type):
    _instances = {}
    def __call__(cls, *args, **kwargs):
        if cls not in cls._instances:
            cls._instances[cls] = super(Singleton, cls).__call__(*args, **kwargs)
        return cls._instances[cls]class Connection(object):
    def __init__(self, name : str, value :float = 0):
        self.__name = name
        self.__value = value
        
        self.__weight = 0
        self.__from_system = ""
        self.__to_systems = set()
    
    @property
    def value(self):
        return self.__value

    @value.setter
    def value(self, value : float):
        self.__value = value
    
    @property
    def name(self):
        return self.__name

    @property
    def weight(self):
        return self.__weight
    
    @weight.setter
    def weight(self, value):
        self.__weight = max(self.__weight, value)
    
    
    @property
    def from_system(self):
        return self.__from_system
    
    @property
    def to_systems(self):
        return self.__to_systems
    
    def add_from_system(self, name):
        self.__from_system = name
        
        
    def add_to_system(self, name):
        self.__to_systems.add(name)
        
from .Connection import Connection
from .ConnectionManager import ConnectionManager

import pandas as pd

import json

class ConnectionDataWrapper(metaclass=Singleton):
    def load_from_csv(cls, path):
        print(f"ConnectionDWrapper. Loading links from <{path}>")
        names = pd.read_csv(path).columns.values
        
        new_names = []
        for name in names:
            if not ConnectionManager().exists(name):
                new_names.append(name)
            ConnectionManager().get_instance(name)
            
        return new_names
    
    
    def to_dict(cls, link_name : str) -> dict:
        if not ConnectionManager().exists(link_name):
            return None
        
        link = ConnectionManager().get_instance(name=link_name)
        
        return dict(name=link_name, 
                    value=link.value,
                    weight=link.weight)
    
    
    def from_dict(cls, link_dict : dict):
        if not all(key in link_dict for key in ('name', 'value', 'weight')):
            return
        
        link = ConnectionManager().get_instance(name=link_dict['name'], 
                                                value=link_dict['value'])
        
        link.weight = link_dict['weight']
        
        return link_dict['name']
    
    
    def from_dicts(cls, link_dict_list : list):
        links = []
        for link_dict in link_dict_list:
            links.append(cls.from_dict(link_dict=link_dict))
        return links
    
    
    def to_json(cls, link_name : str) -> str:
        if not ConnectionManager().exists(link_name):
            return None
        
        return json.dumps(cls.to_dict(link_name))from ..Singleton import Singleton
from .Connection import Connection

import re

class ConnectionManager(metaclass=Singleton):
    __links = {}
    __internal_link_number = 1
    
    def __add(cls, link):
        name = link.name
        if name not in cls.__links:
            # print(f"ConnectionManager. Link added: {name}")
            cls.__links[name] = link
    
    
    def __get(cls, name):
        if name in cls.__links:
            return cls.__links[name]
        return None
    
    
    def clear(cls):
        # print(f"ConnectionManager. Links deleted")
        cls.__links = {}
        cls.__internal_link_number = 1
    
    
    def __set_internal_number(self, n):
        __internal_link_number = max(__internal_link_number, n)
    
    
    def get_instance(cls, name : str, value : float = 0):
        link = cls.__get(name)
        if link is None:
            link = Connection(name, value)
            cls.__add(link)
        return link
    
    
    def get_instances(cls, names = []):
        links = []
        for name in names:
            links.append(cls.get_instance(name))
        return links
    
    
    # Return all keys
    def get_keys(cls):
        return list(cls.__links.keys())
    
    
    #Add specific number of internal links
    def create_internal_connections(cls, link_number : int = 1):
        links = []
        for i in range(link_number):
            name = f"_link{cls.__internal_link_number}"
            links.append(name)
            cls.__internal_link_number += 1
        
        return links
    
    
    def rebase_internal_connections(cls):
        for key in cls.get_keys():
            if '_link' in key:
                possible_values = [int(val) for val in re.findall('\d+', key)]
                n = max(possible_values)
                cls.__internal_link_number = n + 1
    
    
    def set_value(cls, name : str, value : float):
        link = cls.__get(name)
        if link is not None:
            link.value = value
    
    
    def exists(cls, name):
        return cls.__get(name) != None
\end{lstlisting}

}