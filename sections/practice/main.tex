\chapter{Разработка алгоритмов и средств динамической идентификации}
%TODO
 
\section{Описание исследуемого объекта}
%TODO: Добавить описание котельного агрегата, базы данных

Теплоэнергетический объект, представленный в данном исследовании, — это паровой котел типа ГМ-50, разработанный для производства насыщенного или перегретого пара. Этот котел относится к категории паровых котлов с естественной циркуляцией и широко применяется в промышленных, коммунальных и энергетических системах для обеспечения тепловой энергией. 

\begin{figure}[H]
  \begin{center}
    \includegraphics[width=0.75\textwidth]{figures/gm-50.jpg}
  \end{center}
  \caption{Структурная схема парового котла ГМ-50}\label{fig:gm50}
\end{figure}


Для проведения динамической идентификации в работе используется датасет, собранный на основе датчиков, установленных на установке (см. рис. \ref{fig:gm50}), и содержащий более 300 параметров, описывает динамику работы теплоагрегатов, предоставляя возможность для детального анализа производительности, эффективности и эксплуатационных характеристик котла.

\section{Выбор архитектуры нейронных сетей}

Появление методов глубокого обучения привели к появлекнию множества различных архитектур нейронных сетей, в частности их компонентов. Архитектура нейронной сети во многом определяется классом задачи и её правильный выбор необходимо делать исходя из особенностей данных и желаемого результата. Несмотря на широкий спектр задач, которые решают те или иные архитектуры, не все модели подходят для решения задач идентификации.

\subsection{Полносвязная нейронная сеть}

Полносвязанные нейронные сети — это нейронные сети, в которых каждый 
нейрон передает свой выходной сигнал остальным нейронам, в том числе 
и самому себе. 

\begin{figure}[H]
  \centering
    \includegraphics[width=0.9\textwidth]{figures/arch_fully_connected.png}
  \caption{Архитектура полносвязной сети}\label{fig:dense_nn}
\end{figure}

Все входные сигналы подаются всем нейронам, находящимся 
на текущем слое (см. рис. \ref{fig:dense_nn}). Выходными сигналами сети могут быть все или некоторые
выходные сигналы нейронов.

Такие модели могут использоваться для аппроксимации сложных многомерных
функций, представляющих множество целей оптимизации. Часто применяются для
построения суррогатных моделей для оценки функций без необходимости прямого
вычисления.

\subsection{Рекуррентная нейронная сеть}

Рекуррентная нейронная сеть — это тип искусственных нейронных сетей, широко
используемый для обработки последовательных данных и временных рядов. 

\begin{figure}[H]
  \centering
    \includegraphics[width=0.9\textwidth]{figures/arch_rnn.png}
  \caption{Архитектура рекуррентной сети}\label{fig:rnn}
\end{figure}

В отличие от традиционных нейронных сетей, например, многослойных перцептронов, где
обработка данных происходит только в одном направлении, RNN имеют петли (см.
рис. \ref{fig:rnn}). Эти петли позволяют сохранять и использовать информацию из 
предыдущих состояний сети, что делает RNN особенно полезными для задач, где важен 
контекст и зависимость данных во времени. 

В задачах динамической идентификации, где необходимо учитывать не только прямые
связи систем, но также и краевые эффекты, оказывающие влияние на смежные
системы, RNN могут быть полезны благодаря своей способности учитывать контекст
и исторические данные.

\subsection{Сверточная нейронная сеть}

Сверточная нейронная сеть — особый тип нейронной сети, основанный на
полносвязной сети и имеющий как минимум один особый сверточный слой. 

\begin{figure}[H]
  \centering
    \includegraphics[width=0.9\textwidth]{figures/arch_cnn.png}
  \caption{Архитектура сверточной сети}\label{fig:cnn}
\end{figure}

Сверточный слой — нейронный слой, позволяющий производить понижение или повышение
размерности данных. Из-за своей особой архитектуры (см. рис. \ref{fig:cnn}),
сети позволяют эффективно обрабатывать данные с пространственной структурой.

Данная категория нейронных сетей применяется в случаях, если входы системы
имеют пространственную структуру, в которой элементы связанны между собой. Они
применяются для обнаружения признаков или шаблонов, влияющих на общую работу
системы.

Зачастую данный тип нейронных сетей используется в связи с другими
архитектурами, предоставляя им возможности работы с небольшими данными с
выделенными общими признаками.

\subsection{Автокодировочная нейронная сеть}

Автокодировщик — это тип нейронной сети, используемой для обучения эффективного
кодирования данных. Цель автокодировщика — научиться представлять входные
данные в более сжатом и информативном виде, называемом латентным пространством,
и затем восстанавливать оригинальные данные из этого представления. 
\begin{figure}[H]
  \centering
    \includegraphics[width=0.9\textwidth]{figures/arch_autoencoder.png}
  \caption{Архитектура сети-автокодировщика}\label{fig:autoencoder}
\end{figure}

Он состоит из двух основных частей: энкодера и декодера (см. рис.
\ref{fig:autoencoder}). Энкодер преобразует входные данные в сжатое 
скрытое представление (латентное пространство), а декодер восстанавливает 
исходные данные из этого представления. Задача декодера — восстановить данные 
из их скрытого сжатого представления. Данный тип моделей позволяет находить 
компактные представления данных и выявлять скрытые закономерности в них.
В некоторых типах автокодировщиков добавляют вероятностную интерпретацию, что
позволяет моделировать неопределенности в данных и оптимизировать несколько
целей одновременно через латентное пространство.

Архитектурные особенности автокодировщиков не предполагают использования их в
качестве моделей, описывающих системы, однако могут использоваться с целью
повторения помех в данных и приближения их к реальным.

\subsection{Сравнение архитектур}
%TODO: Описание тестового набора

Для того, чтобы сравнить производительность различных архитектур нейронных
сетей выберем из рассмотренного ранее набора данных набор связанных параметров
и произведем для выбранной подсистемы построение цифрового двойника. 

В качестве исследуемых параметров выберем систему с одним входным и выходным
параметром - нагревательный котел. В качестве входного параметра выберем
массовый расход мазута, а выходным температура в котле. Соответственно,
повышение мазута, поступающего на нагрев котла, должно приводить к повышению
температуры внутри котла. 

\begin{figure}[H]
  \centering
    \includegraphics[width=0.85\textwidth]{figures/plots/kotel_temp_mazut.png}
  \caption{Временные характеристики параметров котла}\label{fig:plt:kotel}
\end{figure}

Из временных зависимости входного и выходного параметров (см. рис.
\ref{fig:plt:kotel}) видно, что параметры имеют связь. 

\begin{figure}[H]
  \begin{center}
    \includegraphics[width=0.85\textwidth]{figures/plots/kotel_temp_mazut_rel.png}
  \end{center}
  \caption{Корреляция между параметрами котла}\label{fig:plt:kotel:rel}
\end{figure}

Из построения характеристики между параметрами (см. рис.
\ref{fig:plt:kotel:rel}) видно, что характер зависимости не является
линейным и трудно описываем простыми выражениями. 

\subsubsection{Полносвязная сеть}

Произведем идентификацию с помощью многослойной
нейронной сети. 

\begin{figure}[H]
  \begin{center}
    \includegraphics[width=0.95\textwidth]{figures/tensorflow/dense.png}
  \end{center}
  \caption{Конфигурация полносвязной сети}\label{fig:tf:dense}
\end{figure}

Архитектура сети выбрана исходя условия отсутствия
сужений относительно количества входов и выходов (см.
рис. \ref{fig:tf:dense}). 

Обучение производится на 50 эпохах с учетом
разделения данных на $80\%$ на обучающую выборку и
$20\%$ тестирование и валидацию. 

\begin{figure}[H]
  \begin{center}
    \includegraphics[width=0.95\textwidth]{figures/tensorflow/dense_compare.png}
  \end{center}
  \caption{Сравнение полученной модели полносвязной нейронной сетью}\label{fig:tf:cmp:dense}
\end{figure}

Из полученных результатов (см. рис. \ref{fig:tf:cmp:dense}) можно видеть, что
общая динамика системы повторяется, кроме того, нейтрализуются всплески,
обусловленные шумом. 

\subsubsection{Рекуррентная сеть}

Произведем идентификацию с помощью рекуррентной
нейронной сети. 

\begin{figure}[H]
  \begin{center}
    \includegraphics[width=0.95\textwidth]{figures/tensorflow/rnn.png}
  \end{center}
  \caption{Конфигурация рекуррентной сети}\label{fig:tf:rnn}
\end{figure}

Архитектура сети выбрана исходя условия отсутствия
сужений относительно количества входов и выходов (см.
рис. \ref{fig:tf:rnn}). 

Обучение будем производить с теми же свойствами, что и предыдущую. 

\begin{figure}[H]
  \begin{center}
    \includegraphics[width=0.95\textwidth]{figures/tensorflow/rnn_compare.png}
  \end{center}
  \caption{Сравнение полученной модели рекуррентной нейронной
  сетью}\label{fig:tf:cmp:rnn}
\end{figure}

Из полученных результатов (см. рис. \ref{fig:tf:cmp:rnn}) можно видеть, что
общая динамика системы повторяется, кроме того, нейтрализуются всплески,
обусловленные шумом, однако выход сети перевернут. 

\begin{figure}[H]
  \begin{center}
    \includegraphics[width=0.95\textwidth]{figures/tensorflow/rnn_compare_reversed.png}
  \end{center}
  \caption{Сравнение полученной модели рекуррентной нейронной
  сетью}\label{fig:tf:cmp:rnn:reversed}
\end{figure}

С учетом переворота графика получаем, что метод имеет меньшую статическую
ошибку нежели полносвязная сеть (см. рис. \ref{fig:tf:cmp:rnn:reversed}).

Однако, реккурентная сеть позволяет получать только следующие значения,
основываясь на последовательностях. Для идентификации, основывающейся на
фактических параметрах модели лучшен использовать полносвязную сеть, наиболее подробно повторяющую динамику системы. 

\section{Декомпозиционный метод идентификации нейронными сетями}
%TODO: Описание предлагаемого метода
Большинство методов идентификации предполагают создание общей модели в виде
<<черного ящика>>, однако для многих больших систем такое моделирование, в
случае наличия дополнительных данных о внутренних сигналах, может быть
неэффективно. Причиной для этого может служить упущением данных о внутренней
структуре или подсистемах, взаимодействующих друг с другом и оказывающих
влияние на общий результат. Кроме того, зачастую необходимо учитывать значение
параметров подсистем при определенных режимах работы.

Точность идентификации во многом зависит от сложности рассматриваемой системы.
Малые системы дают более точный результат при меньшей сложности создания модели
или обучении нейронной сети. Из этого можно сделать вывод, что точность
построения цифрового двойника во многом связана с размером моделируемой
системы, чем меньше система, тем проще её описать. 

В связи с этим, предлагается непараметрический метод идентификации с
использованием связанного множества нейронных сетей, представляющих подсистемы
общей системы. Каждая рассматриваемая нейронная сеть иммитирует поведение
каждой своей подсистемы. При этом всём важно обеспечить связь между
подсистемами для моделирования все системы в целом. Для этого необходимо
соотвествтующее программное обеспечение, которое бы позволяло производить
полный цикл действий по динамической идентификации предложенным методом. 

\section{Программное обеспечение динамической идентификации}
%TODO: Привести информацию о наличии других ПО для моделирования
%       и то, что зачастую сложные методы поставляются с использованием замкнутого цикла ПО

Разработка удобного графического интерфейса, наиболее полно расрывающего
возможности предлагаемого решения для пользователья, сложный процесс, требующий
точного и скурпулёзного проектирования. В рамках разработки важными этапами
являются:

\begin{itemize}
  \item Выделение требований к разрабатываемой системе;
  \item Определение паттернов и структур данных;
  \item Реализация интерфейса;
  \item Оценка работы системы.
\end{itemize}

\subsection{Требования к системе}
%TODO: Привести требования к предлагаемому ПО

Выделение требований один из наиважнейших этапов, который выделяет необходимые требования и возможности будущего приложения. В рамках разработки приложения для реализации предложенного метода идентификации были выделены следующие требования:

\begin{itemize}
  \item Создавать или загружать данные о сигналах;
  \item Конфигурировать системы с заданным именем, входными и выходными
    сигналами;
  \item Объединять созданные подсистемы в единую связанную систему в
    соответствии с параметрами, определенными пользователем;
  \item Задавать параметры обучения для каждой системы;
  \item Отображать созданную систему в виде связанного графа;
  \item Позволять вручную вводить значения для каждого из сигналов;
  \item Сохранять параметры систем и их иерархию в виде файлов.
\end{itemize}

Все выше перечисленные возможности должны быть реализованы в рамках одного приложения.


\subsection{Структуры данных}
%TODO: Какие структуры данных, их иерархия, паттерны использовались
Основой любого приложения являются данные. Правильный выбор абстракций и определение необходимых структур данных на начальном этапе разработке играет большую роль, определяющую такие свойства приложения как масштабируемость приложения с возможностью расширения функциональных возможностей. 

Кроме того, важно выбрать паттерны проектирования, позволяющие работать с большим количеством объектов. 

В рамках анализа базовых компонентов было выделено 2 основных базовых класса: 

\begin{itemize}
  \item Связь (Connection);
  \item Система (System). 
\end{itemize}

\subsubsection{Класс <<Связь>>}

Класс связи (Connection) предназначен для реализации связывания подсистем и процессов, протекающих между системами. 

Класс также обеспечивает возможности передачи сведений от одной системы к другой (значений сигналов). 

Каждый сигнал уникален и не повторяет по имени никакой другой.

\subsubsection{Класс <<Система>>}

Класс системы (System) предназначен для абстракции данных, связанных с
какой-либо подсистемой. Он содержит информацию о сигналах, являющихся входными
или выходными. Также, он содержит информацию о модели нейронной сети,
аппроксимирующей её поведение. 

Каждая система должна удовлетворять критерию уникальности и не быть дубликатом
другой системы. 

\subsubsection{Менеджеры классов}

Для корректного использования объектов систем и сигналов
реализованы классы, управляющие созданием и изменением объектов данного
класса – менеджеры. 

Каждый класс-менеджер, ConnectionManager и SystemManager, реализует паттерны
Singleton и Abstract Factory - в рамках всего приложения менеджеры уникальны и
учитывают созданные в текущей сессии системы и связи. 

Также, менеджеры обеспечивают учет связей между компонентами, помня какой
сигнал с какой системой связан, позволяя получать граф связей без необходимости
обхода всей системы.

\subsubsection{Вспомогательные классы}

Для поддержания абстракций реализован класс аппроксиматор, реализуйющий
создание и обучение нейронной сети для конкретной заданной системы -
SystemNeuralModel. Любые действия, связанные с идентификационной моделью,
включая создание, обучение или моделирование, выполняется через него. 

\begin{figure}[H]
  \begin{center}
    \includegraphics[width=0.95\textwidth]{figures/basics_relations.png}
  \end{center}
  \caption{Взаимодействие базовых компонент системы}\label{fig:basics:components}
\end{figure}

Каждый из классов (см. рис. \ref{fig:basics:components}) работает в связи с
другими и реализует базовый функционал для конфигурации исследуемых сигналов и
систем.

\subsection{Архитектура приложения}
%TODO: Привести список компонентов, их назначение и связь между собой

Программное обечпечение реализует возможности проведения процесса динамической
идентификации, обеспечивая следующих возможности:
\begin{itemize}
  \item Компановка структуры сигналов и систем;
  \item Обучение моделей на базе нейронных сетей;
  \item Моделирование процессов системы.
\end{itemize}

Для обеспечения выполнения перечисленных возможностей, программное обеспечение
состоит из нескольких ключевых компонентов, обеспечивающих функциональность
динамической идентификации. 

\begin{figure}[H]
  \begin{center}
    \includegraphics[width=0.95\textwidth]{figures/modules/relations.png}
  \end{center}
  \caption{Модульные связи приложения}\label{fig:modules:relation}
\end{figure}

В частности, программа состоит из двух компонент (см. рис.
\ref{fig:modules:relation}) - модуля представления данных и модуля графического
интерфейса. 

Система представления данных содержит модуль конфигурации и нейросетевой
модуль, реализующий логику поведения систем и их компановку и их идентификацию.

Графический интерфейс реализует модули, позволяющие взаимодействовать с
компонентами системы представления и в наглядной форме производить
моделирование. 

\subsection{Используемые формы}
%TODO: Перечисление форм и их назначение, какие возможности предоставляют

Графический интерфейс реализует элементы управления и отображения, позволяющие
удобно производить процесс динамической идентификации. Приложение состоит из
четырех смежных модулей, охватывающих различные области управления и
отображения информации:

\begin{itemize}
  \item Модуль загрузки данных;
  \item Модуль конфигурации;
  \item Модуль настройки обучения;
  \item Модуль просмотра и моделирования.
\end{itemize}

\subsubsection{Модуль загрузки данных}

Модель загрузки данных отвечает за загрузку и предварительную обработку
экспериментальных данных, необходимых для обучения нейронных сетей каждой
подсистемы. 

\begin{figure}[H]
  \begin{center}
    \includegraphics[width=0.95\textwidth]{figures/modules/loader.png}
  \end{center}
  \caption{Форма загрузки и обработки данных}\label{fig:forms:loader}
\end{figure}

\subsubsection{Модуль конфигурации}

Модуль компановки систем и их связей позволяет пользователю определять
структуру промышленной установки, указывая конфигурацию подсистемы и сигналы,
описывающие поведение системы. 

\begin{figure}[H]
  \begin{center}
    \includegraphics[width=0.95\textwidth]{figures/modules/editor.png}
  \end{center}
  \caption{Форма конфигурации систем и связей}\label{fig:forms:editor}
\end{figure}

\subsubsection{Модуль настройки обучения}

Модуль обучения нейронных сетей реализует возможность задания параметров
обучения нейросети, отслеживания процесса обучения нейронных сетей каждой
подсистемы на загруженных данных, а также получение отчета об обучении. 

\begin{figure}[H]
  \begin{center}
    \includegraphics[width=0.95\textwidth]{figures/modules/neural.png}
  \end{center}
  \caption{Форма настройки обучения и структуры нейронных
  сетей}\label{fig:forms:neural}
\end{figure}

\subsubsection{Модуль просмотра и моделирования}

Модель просмотра скомпонованной системы предоставляет визуальное представление
структуры установки с выделением подсистем и связей между ними, а также
позволяет пользователю проводить симуляцию поведения идентифицированной системы
на основе обученных нейронных сетей. Взаимодействие между этими модулями
осуществляется посредством передачи данных через связи подсистем с учетом
заданной конфигурации.

\begin{figure}[H]
  \begin{center}
    \includegraphics[width=0.95\textwidth]{figures/modules/modelling.png}
  \end{center}
  \caption{Форма просмотра и моделирования}\label{fig:forms:viewer}
\end{figure}

Моделирование параметров систем может происходить как от начальных входов всей
системы, так и от частных входов подсистем. Изменение величин на выходах
моделей происходит последовательно и затрагивает все системы, которые связаны с
той, от которой произошли изменения.
Полный перерасчет всей модели будет происходить даже в случае задания значения
промежуточных сигналов, так как они не могут рассматриваться как независимые
сигналы. К независимым сигналам можно относить только сигналы на входе общей
системы.

\section{Архитектура связей нейронных сетей}
%TODO: Описать какие модели используются при обучении для описания каждой подсистемы и системы в общем

Компановка нейронных сетей является важной частью предлагаемого метода. В связи с рассмотренными особенностями нейронных сетей в задачах динамической идентификации, нейросетевые модели каждой подсистемы состоят из двух компонент - полносвязной сети и автокодировщика. 

\begin{figure}[H]
  \begin{center}
    \includegraphics[width=0.95\textwidth]{figures/nn_system.png}
  \end{center}
  \caption{Архитектура нейронных сетей для подсистемы}\label{fig:nn:system}
\end{figure}

Полносвязная сеть (см. рис. \ref{fig:nn:system}) предназначена для обеспечения воспроизведения динамики системы и выполнения преобразования входных велечин в выходные. 

Автокодировщик выступает как модель помех подсистемы. Автокодировщик стоит на выходе системы и учится воспроизводить зашумление выходного сигнала. Автокодировщик был выбран исходя из его способности вопроизводить входной сигнал, а также из-за возможности нейтрализации резких всплесков. Благодаря этому возможно отсеять всплекси, обусловленные на источник измерения в обучаеющих данных. 


Для общей системы также используется сеть автокодирощик, реализующий аппроксиматор шума всей системы (см. рис. ниже).

\begin{figure}[H]
  \begin{center}
    \includegraphics[width=0.95\textwidth]{figures/nn_full.png}
  \end{center}
  \caption{Архитектура нейронных сетей для подсистемы}\label{fig:nn:full}
\end{figure}

\subsection{Порядок обучения}

Для обеспечения наилучшего качества обучения каждой из нейронных сетей,
обучение производится по определенному порядку. 

Первоначально обучаются полносвязные сети каждой подсистемы. После их обучения,
обучаются автокодирующие сети внутри подсистем. После обучения каждой из
подсистем, производится обучение модели внешнего шума. 

\begin{figure}[H]
  \begin{center}
    \includegraphics[width=0.95\textwidth]{figures/nn_full_learning.png}
  \end{center}
  \caption{Последовательность обучения нейросетевых
  моделей}\label{fig:nn:learning}
\end{figure}



\section{Оценка работы}
%TODO: Произвести оценку моделирования системы

Для оценки работы предлагаемого метода, произведем идентификацию и моделирование на сегрегированной модели, на базе компонентов энергетического объекта ГМ-50. 

\begin{table}[H]
\caption{Описание компонентов подсистем моделирования}\label{tab:subsystems}
\begin{tabular}{|p{0.25\textwidth}|p{0.34\textwidth}|p{0.34\textwidth}|}
    \hline
    Подсистема  & Входы & Выходы \\
    \hline 
    Пароводяной котел & Расход мазута в котёл;  & Давление воды в котёл; \\
                      & Температура мазута к котлу. & Давление перегретого пара от котла. \\
    \hline 
    Мазутный насос    & Давление мазута на котел & Уровень мазута в резервуаре \\
    \hline
    Нагревательный котел & Давление воды в котёл; & Температура мазута в резервуаре средняя;\\
                         & Давление перегретого пара от котла; & Давление пара от котла ($log$).\\
                         & Уровень мазута в резервуаре. & \\
    \hline
\end{tabular}
\end{table}

Описание каждой подсистемы приведено в таблице \ref{tab:subsystems}. Структурную схему экспериментальной установки можно видеть на рисунке ниже. 

\begin{figure}[H]
  \begin{center}
    \includegraphics[width=0.95\textwidth]{figures/subsystem_diagram.png}
  \end{center}
  \caption{Структурная схема экспериментальной установки}\label{fig:subsystem_diagram}
\end{figure}

%TODO
\begin{figure}[H]
  \begin{center}
    \includegraphics[width=0.95\textwidth]{figures/modules/loader.png}
  \end{center}
  \caption{Загруженные данные о работе системы}\label{fig:test:loaded_data}
\end{figure}

Загрузив данные о работе системы (см. рис. \ref{fig:test:loaded_data}), определим конфигурацию системы, в частности названия, а также список входов и выходов на вкладке конфигурации (см. рис. ниже).

%TODO
\begin{figure}[H]
  \begin{center}
    \includegraphics[width=0.95\textwidth]{figures/modules/editor.png}
  \end{center}
  \caption{Сконфигурированная экспериметнальная система}\label{fig:test:configured_system}
\end{figure}

Проверить корректность конфигуарции можно во вкладке обзора и моделирования (см. рис. ниже). 

%TODO
\begin{figure}[H]
  \begin{center}
    \includegraphics[width=0.95\textwidth]{figures/modules/modelling.png}
  \end{center}
  \caption{Вид экспериметнальной системы в окне просмотра}\label{fig:test:preview}
\end{figure}

Запустим обучение для каждой подсистемы на 20 эпохах в окне обучения (см. рис. ниже). 

%TODO
\begin{figure}[H]
  \begin{center}
    \includegraphics[width=0.95\textwidth]{figures/subsystem_diagram.png}
  \end{center}
  \caption{Экспериментнальная конфигурация идентификационных моделей}\label{fig:test:neural_form}
\end{figure}

Обученние моделей происходит в соответствии с алгоритмом, описанном в технической часть. Результаты обучения и валидации нейронных сетей можно видеть на рисунке \ref{fig:test:plot:learning}.

%TODO Сделать табличку
\begin{figure}[H]
  \centering
  \subfigure[]{\includegraphics[width=0.49\textwidth]{figures/test/sys1.png}}
  \subfigure[]{\includegraphics[width=0.49\textwidth]{figures/test/sys2.png}}
  \subfigure[]{\includegraphics[width=0.49\textwidth]{figures/test/sys3.png}}
  \caption{Результаты ошибки при обучении и проверки\\(a) Пароводяной котел (b) Мазутный насос (c) Нагревательный котел}\label{fig:test:plot:learning}
\end{figure}

\begin{figure}[H]
  \centering
  \subfigure[]{\includegraphics[width=0.49\textwidth]{figures/test/sys1_noise.png}}
  \subfigure[]{\includegraphics[width=0.49\textwidth]{figures/test/sys2_noise.png}}
  \subfigure[]{\includegraphics[width=0.49\textwidth]{figures/test/sys3_noise.png}}
  \subfigure[]{\includegraphics[width=0.49\textwidth]{figures/test/all_noise.png}}
  \caption{Результаты ошибки автокодировщика при обучении и проверки\\(a) Пароводяной котел (b) Мазутный насос (c) Нагревательный котел (d) Все системы}\label{fig:test:plot:learning:encoder}
\end{figure}

Как можно видеть, обучение каждой из нейронных сетей (полносвязной и автокодирующей) прошло успешно. Проверка работы каждой подсистемы над воспроизведением всей подсистемы также показывает, что подлсистемы успешно обучились воспроизводить внутренние процессы каждого компонента всей системы (см. рис. ниже).

%TODO: Табличка с графиками подсистем
\begin{figure}[H]
  \centering
  \subfigure[]{\includegraphics[width=0.49\textwidth]{figures/test/sys1_res.png}}
  \subfigure[]{\includegraphics[width=0.49\textwidth]{figures/test/sys2_res.png}}
  \subfigure[]{\includegraphics[width=0.49\textwidth]{figures/test/sys3_res.png}}
  \caption{Воспроизведение работы подсистем идентификационных нейронных моделей\\(a) Пароводяной котел (b) Мазутный насос (c) Нагревательный котел}\label{fig:test:plot:subsystems}
\end{figure}

Также, произведем общее моделирование с учетом передачи сигналов между сетями. 

%TODO: Табличка с графиками подсистем
\begin{figure}[H]
  \begin{center}
    \includegraphics[width=0.95\textwidth]{figures/test/all_res.png}
  \end{center}
  \caption{Воспроизведение работы целой системы идентификационных нейронных моделей}\label{fig:test:plot:system}
\end{figure}

Из графика полученного сигнала при моделировании с использованием обученнй модели (см. рис. \ref{fig:test:plot:system}) можно видеть, что предложенный метод воспроизводит динамику системы с большой точностью, а также с должной точностью воспроизводит значения сигналов.
