\Conclusion % заключение к отчёту

В рамках работы был предложен метод проведения динамической идентификации с
использованием декомпозиции сложных систем и аппроксимации их поведения с
использованием нейронных сетей. В рамках обоснования применимости были
рассмотрены традиционные методы и было установлена их сложность в задачах
непараметрической оптимизаци при наличии лишь эксплуатационных данных о системе.
Также, для предложенного метода был разработан программный продукт
представляющий собой инструмент для динамической идентификации. Программа
включает в себя оконный интерфейс, охватывающий все этапы работы по
идентификации, от загрузки данных до моделирования процессов. 

В рамках применения метода и его оценки было установлено, что система показывает
результаты лучше, чем при использовании обычного метода нейронных сетей в виде
<<общего черного ящика>>. Рассмотренный метод обученной на экспериментальных
данных, демонстрирует потенциал для упрощения процесса моделирования и повышения
точности прогнозирования поведения сложных многокомпонентных систем. 


%%% Local Variables: %% mode: latex %% TeX-master: "rpz" %% End: 

