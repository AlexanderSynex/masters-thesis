%% Преамбула TeX-файла

% 1. Стиль и язык
\documentclass[utf8x, 14pt]{G7-32} % Стиль (по умолчанию будет 14pt)

% Остальные стандартные настройки убраны в preamble.inc.tex.
% \include{preamble.inc}

% Настройки листингов.
%\ifPDFTeX
%\include{listings.inc}
%\else
%\usepackage{local-minted}
%\fi

% Доп определения

\usepackage{pgfplots}
% \usepackage{subcaption}


\pgfplotsset{compat=newest} % <-- optional in preamble

\definecolor{m_blue}{rgb}{0, 0.4470, 0.7410}
\definecolor{m_red}{rgb}{0.8500, 0.3250, 0.0980}
\definecolor{m_yellow}{rgb}{0.9290, 0.6940, 0.1250}
\definecolor{m_purple}{rgb}{0.4940, 0.1840, 0.5560}
\definecolor{m_green}{rgb}{0.4660, 0.6740, 0.1880}
\definecolor{m_cyan}{rgb}{0.3010, 0.7450, 0.9330}

\usetikzlibrary{backgrounds,automata}

% Полезные макросы листингов.
%\include{macros.inc}
% Стиль титульного листа и заголовки
% \include{00-title}


\begin{document}

\frontmatter % выключает нумерацию ВСЕГО; здесь начинаются ненумерованные главы: реферат, введение, глоссарий, сокращения и прочее.

\maketitle %создает титульную страницу





%\listoffigures                         % Список рисунков

%\listoftables                          % Список таблиц

%\NormRefs % Нормативные ссылки
% Команды \breakingbeforechapters и \nonbreakingbeforechapters
% управляют разрывом страницы перед главами.
% По-умолчанию страница разрывается.

% \nobreakingbeforechapters
% \breakingbeforechapters

\begin{titlepage}
  \begin{center}

    Титульный лист 
    
    Список исполнителей 
   
  \end{center}
\end{titlepage}



\tableofcontents

\printnomenclature % Автоматический список сокращений

% Также можно использовать \Referat, как в оригинале
\begin{Referat}

    Отчет содержит \pageref{LastPage}\,стр.%
    \ifnum \totfig >0 , \totfig~рис.%
    \fi
    \ifnum \tottab >0 , \tottab~табл.%
    \fi
    %
    \ifnum \totbib >0 , \totbib~источн.%
    \fi
    %
    \ifnum \totapp >0 , \totapp~прил.%
    \else
    .%
    \fi

    Ключевые слова: идентификация систем, нейронные сети, моделирование
    процессов, программные средства идентификации, вычислительные системы.

    Тема выпускной квалификационной работы: <<Разработка алгоритмов и программ
    динамической идентификации энергетического объекта>>.

    Данная работа посвящена разработке и применению метода динамической
    идентификации сложных промышленных систем с использованием декомпозиционного
    подхода и нейронных сетей. Основной целью исследования является создание
    эффективного инструмента для построения цифровых двойников многокомпонентных
    систем, обеспечивающего высокую точность моделирования и упрощение процесса
    идентификации. Работа охватывает теоретические основы, разработку
    программного обеспечения и практическую апробацию предложенного метода на
    примере теплоэнергетического объекта — парового котла ГМ-50. 
    \nocite{*}

\end{Referat}

%%% Local Variables: %% mode: latex %% TeX-master: "rpz" %% End: 


\begin{ReferatEng}

    On \pageref{LastPage}\,pages%
    \ifnum \totfig >0 , \totfig~figures%
    \fi
    \ifnum \tottab >0 , \tottab~tables%
    \fi
    %
    \ifnum \totbib >0 , \totbib~sources%
    \fi
    %
    \ifnum \totapp >0 , \totapp~appendices%
    \else
    .%
    \fi

    Keywords: system identification, neural networks, process modeling,
    identification software tools, computational systems.

    The subject of the graduate qualification work is <<Development of
    algorithms and software for dynamic identification of an energy facility>>.

    This study focuses on the development and application of a dynamic
    identification method for complex industrial systems using a decompositional
    approach and neural networks. The primary objective of the research is to
    create an effective tool for building digital twins of multicomponent
    systems, ensuring high modeling accuracy and simplifying the identification
    process. The work covers theoretical foundations, software development, and
    practical validation of the proposed method, demonstrated on a thermal power
    facility—the GM-50 steam boiler.
    \nocite{*}

\end{ReferatEng}

\mainmatter % это включает нумерацию глав и секций в документе ниже

%\include{20-analysis}
%\include{30-design}
% % Теоретическая часть
% \chapter{Динамическая идентификация систем}

Динамическая идентификация представляет собой процесс построения математических
моделей динамических систем на основе измерений их входных и выходных сигналов.
В динамической системе значения выходных сигналов зависят как от текущих
значений входных сигналов, так и от предшествующего поведения системы
\cite{bb:identification}. Типичные этапы динамической идентификации включают
следующие этапы: 

\begin{enumerate}
  \item Определение и выделение исследуемых параметров системы;
  \item Формирование набора данных о работе исследуемой системы;
  \item Выбор метода идентификации;
  \item Разработка модели
  \item Оценка параметров и валидация полученой модели.
\end{enumerate}

Основная задача динамической идентификации состоит в выявлении неявных
закономерностей между параметрами модели, которые бы наилучшим образом
описывали наблюдаемое поведение системы. Такая модель может быть использована
для прогнозирования поведения системы при изменении входных параметров или для
оптимизации работы установки. 

Раздел математического моделирования, специализирующего на динамической
идентификации имеет широкий инструментарий и большое количество методов
построения цифровых двойников. В частности, все методы можно разделить на
следующие большие группы:

\begin{itemize}
  \item Параметрические;
  \item Непараметрические;
  \item Гибридные.
\end{itemize}

Параметрические методы включают в себя методы, предполагающие наличие
информации о структуре исследуемой модели. Для данного класса методов задача
сводится к определению параметров соотношений, описывающих модели. Данный класс
методов требует серьезного методологического анализа и учет мельчайших
особенностей работы системы. Данный класс методов зачастую сложно применять в
ситуациях, где система имеет большое количество компонентов и подвержена
внешним воздействиям, однако такие модели имеют наибольшую точность и наиболее
широко обхватывают все процессы, которым подвержена моделируемая система.
Непараметрические методы, в отличии от параметрических,
не требуют априорного знания о структуре модели и основывается лишь на данных о
работе системы. Данный класс методов основывается на построении
аппроксимационных моделей системы для определенных режимов работы.
Непараметрические методы применяются для описания сложных комплексных систем
для которых не требуется точности, но нужно общее поведение системы. Гибридные
методы же сочетают в себе особенности параметрических и непараметрических. 

С широким развитием интеллектуальных моделей, в частности, основанных на базе технологии нейронных сетей, классификацию методов можно разделить на традиционные и интеллектуальные методы.

\section{Традиционные методы}

\subsection{Линейные модели}

\subsection{Нелинейные модели}

\section{Интеллектуальные методы}

\subsection{Алгоритмы машинного обучения}

\subsection{Нейронные сети}



% % Техническая часть
% \chapter{Программные средства технологических промышленных систем}

\section{Работа с нейронными сетями}

\subsection{Язык программирования}

\subsection{Фреймворк}

\section{Инструменты создания оконных интерфейсов}


% Практическая часть
\chapter{Разработка алгоритмов и средств динамической идентификации}

\section{Декомпозиционный метод идентификации}

\section{Программное средство динамической идентификации}

%TODO
\subsection{}


% Сравнительная часть


%\include{50-research}
% \include{60-economics}
% \include{70-bzd}

\backmatter %% Здесь заканчивается нумерованная часть документа и начинаются ссылки и

\Conclusion % заключение к отчёту

В рамках работы был предложен метод проведения динамической идентификации с использованием декомпозиции сложных систем и аппроксимации их поведения с использованием нейронных сетей. В рамках обоснования применимости были рассмотрены традиционные методы и было установлена их сложность в задачах непараметрической оптимизаци при наличии лишь эксплуатационных данных о системе. 
Также, для предложенного метода был разработан программный продукт представляющий собой инструмент для динамической идентификации. Программа включает в себя оконный интерфейс, охватывающий все этапы работы по идентификации, от загрузки данных до моделирования процессов. 

В рамках применения метода и его оценки было установлено, что система показывает результаты лучше, чем при использовании обычного метода нейронных сетей в виде <<общего черного ящика>>. 
Рассмотренный метод обученной на экспериментальных данных, демонстрирует потенциал для упрощения процесса моделирования и повышения точности прогнозирования поведения сложных многокомпонентных систем. 


%%% Local Variables: 
%%% mode: latex
%%% TeX-master: "rpz"
%%% End: 

%% заключение


% Список литературы

% % Список литературы при помощи BibTeX
% Юзать так:
%
% pdflatex rpz
% bibtex rpz
% pdflatex rpz

\bibliographystyle{ugost2008}
\bibliography{sections/literature}

%%% Local Variables: 
%%% mode: latex
%%% TeX-master: "rpz"
%%% End:



% \appendix   % Тут идут приложения
%
% \include{90-appendix1}
%
% \include{91-appendix2}

\end{document}

%%% Local Variables:
%%% mode: latex
%%% TeX-master: t
%%% End:
